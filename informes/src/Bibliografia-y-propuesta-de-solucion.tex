\documentclass[final]{article}
\usepackage[numbers,sort&compress]{natbib} % formato de citas
\usepackage{neurips}            % estilo de la conferencia NeurIPS
\usepackage[spanish]{babel}     % definiciones para el español
\usepackage{graphicx}           % para insertar figuras
\usepackage{amsmath,amssymb}    % más símbolos útiles 
%\usepackage{todonotes}          % notas en el documento
\usepackage[hidelinks]{hyperref}% para los links
\usepackage[ruled,vlined,lined,linesnumbered,onelanguage]{algorithm2e} % algoritmos
\usepackage{multirow}           % algunas mejoras para las tablas
\usepackage{booktabs}


\begin{document}
\def\tablename{Tabla}          % (sino les pone Cuadro)

\title{Proyecto final creativo:\\
    Bibliografía y propuesta de solución}

\author{G. Cagnola, F. G. Hergenreder y L. Saurin \\
        Facultad de Ingeniería y Ciencias Hídricas,
        Universidad Nacional del Litoral \\
        Instituto de Investigación en Señales, Sistemas e Inteligencia Computacional, UNL-CONICET}

\maketitle

\begin{abstract}
Muchas ovejas son criadas por pastoreo, que consiste en dejarlas desplazarce libremente y consumir vegetación silveste.
Esto las expone a predadores, como los lobos. Aquí entra en acción el perro pastor, cuyo objetivo es el de proteger el rebaño
de posibles amenazas, y guiar y controlar su movimiento durante los desplazamientos.\\
El objetivo de este proyecto es modelar el comportamiento de un rebaño de ovejas y una manada de lobos (dinámica predador-presa)
para luego entrenar perror pastores para el cuidado y control del desplazamiento del rebaño.
\end{abstract}

%==========================================================
%==========================================================
\section{Tópicos involucrados (no se que titulo poner, marco teórico?)}

\subsection{Sistemas Multiagente (MAS - Multi-agent Systems)}
Necesario para el modelado de múltiples entidaded autónomas, la comunicación, coordinación y resolución de conflictos entre agentes y
lograr a partir de interacciones locales un comportamiento emergente de grupo.

\subsection{Algoritmos coevolutivos (Coevolutionary algorithms)}
Para lograr evolucionar múltiples poblaciones simultaneamente, necesitamos hacer uso de algorítmos genéticos y estrategias evolutivas.
El comportamiento de cada especie se adapta en respuesta a las otras, llevando una dinámica de predador-presa o cooperación.

\subsection{Aprendizaje por refuerzo (Reinforcement Learning - RL)}
Las ovejas aprenden a maximizar la supervivencia, mientras los perros aprenden a maximizar la supervivencia de las ovejas y a lograr
guiarlas a un objetivo, y los lobos aprenden a cazar ovejas

\subsection{Inteligencia de enjambre (Swarm Intelligence)}
El movimiento de rebaño y la evasión coordinada. Esta relacionado con el modelo Boids, PSO y ACO.

\section{Búsqueda bibliográfica}

\subsection{Palabras clave}
\begin{itemize}
    \item Frases y conceptos para buscar papers:
    \item Coevolutionary multi-agent systems
    \item Predator-prey coevolution simulation
    \item Emergent shepherding behavior
    \item Multi-agent reinforcement learning shepherd problem
    \item Evolutionary robotics shepherding
    \item Swarm intelligence in predator-prey environments
    \item Artificial life ecosystems simulation
    \item Neuroevolution of cooperative behavior
    \item Behavioral emergence in multi-agent reinforcement learning
    \item Competitive coevolution dynamics
\end{itemize}

\subsection{Bibliografía encontrada}

\begin{itemize}
    \item Multi-Agent Reinforcement Learning for Shepherding, Abhishek Sankar - 2021
    \item \citet{NapolitanoShepherding}
    \item \citet{}
\end{itemize}

\bibliographystyle{unsrtnat}
\bibliography{refs}
\end{document}

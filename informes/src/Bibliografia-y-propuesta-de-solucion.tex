\documentclass[final]{article}
\usepackage[numbers,sort&compress]{natbib} % formato de citas
\usepackage{neurips}            % estilo de la conferencia NeurIPS
\usepackage[spanish]{babel}     % definiciones para el español
\usepackage{graphicx}           % para insertar figuras
\usepackage{amsmath,amssymb}    % más símbolos útiles 
%\usepackage{todonotes}          % notas en el documento
\usepackage[hidelinks]{hyperref}% para los links
\usepackage[ruled,vlined,lined,linesnumbered,onelanguage]{algorithm2e} % algoritmos
\usepackage{multirow}           % algunas mejoras para las tablas
\usepackage{booktabs}


\begin{document}
\def\tablename{Tabla}          % (sino les pone Cuadro)

\title{Proyecto final creativo:\\
    Bibliografía y propuesta de solución}

\author{G. Cagnola, F. G. Hergenreder y L. Saurin \\
        Facultad de Ingeniería y Ciencias Hídricas,
        Universidad Nacional del Litoral \\
        Instituto de Investigación en Señales, Sistemas e Inteligencia Computacional, UNL-CONICET}

\maketitle

\begin{abstract}
Muchos rebaños de ovejas son criados por pastoreo, que consiste en dejarlas desplazarse libremente y consumir vegetación silveste.
Aquí entra en acción el perro pastor, cuyo objetivo es el de proteger el rebaño
de posibles amenazas, y guiar y controlar su movimiento durante los desplazamientos.\\
El objetivo de este proyecto es el entrenamiento de agentes "pastores", con la finalidad explorar estrategias que podrían usarse para la evacuación de multitudes.
\end{abstract}

\section{Bibliografía seleccionada}
En la búsqueda de estrategias que permitan modelar comportamientos colectivos complejos dentro de sistemas multiagente, el problema del pastoreo
(shepherding problem) se ha convertido en un caso de estudio fundamental para el análisis y la síntesis de comportamientos emergentes coordinados.
Este problema consiste en el diseño de uno o varios agentes “pastores” que deben guiar a un conjunto de agentes “rebaño” hacia una región objetivo
o mantenerlos agrupados, sin control directo sobre cada individuo. Su interés radica no solo en las aplicaciones prácticas —como la robótica cooperativa,
la gestión de drones o el control de multitudes—, sino también en su valor como entorno de simulación para estudiar la autoorganización y cooperación
emergente en sistemas distribuidos.

Años más tarde, \citet{Strombom} abordaron específicamente el problema del pastoreo en su trabajo “Solving the shepherding problem: heuristics
for herding autonomous, interacting agents”. Allí presentan un modelo heurístico que descompone el comportamiento del pastor en dos subcomportamientos
principales: collecting (reunir al rebaño disperso) y driving (guiar al grupo hacia el objetivo). Este enfoque heurístico logró replicar comportamientos
observados en animales reales y sentó una base sólida para aproximaciones posteriores que buscan automatizar el aprendizaje de estas estrategias mediante
técnicas de inteligencia computacional.

Más recientemente, el avance del aprendizaje por refuerzo (Reinforcement Learning, RL) ha impulsado el desarrollo de soluciones más adaptativas y menos
dependientes de heurísticas predefinidas. En “Emergent Cooperative Strategies for Multi-Agent Shepherding via Reinforcement Learning”, \citet{NapolitanoShepherding} exploran cómo los agentes pastores pueden aprender comportamientos cooperativos de manera emergente a través de un esquema de aprendizaje por refuerzo multiagente (Multi-Agent Reinforcement Learning, MARL). Su estudio demuestra que se pueden descubrir estrategia de cooperación espontánea sin programar explícitamente roles, con una adecuada definición de recompensas.

En una línea complementaria, Covone et al. (2025) presentan “Hierarchical Policy-Gradient Reinforcement Learning for Multi-Agent Shepherding Control
of Non-Cohesive Targets” [2], donde proponen una arquitectura jerárquica basada en gradiente de políticas. Este enfoque introduce un control de alto
nivel encargado de decidir entre las fases de collecting y driving, mientras que un nivel inferior ejecuta políticas locales aprendidas. Los autores
muestran que esta estructura jerárquica permite un control más eficiente y estable, especialmente en escenarios con rebaños no cohesivos o entornos dinámicos.

En conjunto, estos trabajos delinean una evolución conceptual del problema del pastoreo: desde las primeras formulaciones basadas en reglas locales y
simulaciones heurísticas, hacia modelos de aprendizaje emergente cooperativo que aprovechan el poder del aprendizaje por refuerzo para lograr
comportamientos más realistas, escalables y generalizables. El presente proyecto se enmarca en esta línea, buscando explorar nuevas formas de entrenamiento
y evaluación de sistemas pastores y rebaños mediante técnicas de inteligencia computacional y simulación multiagente.

\section{Propuesta de solución}
En conjunto, estos trabajos delinean una evolución conceptual del problema del pastoreo: desde las primeras formulaciones basadas en reglas locales y
simulaciones heurísticas, hacia modelos de aprendizaje emergente cooperativo que aprovechan el poder del aprendizaje por refuerzo para lograr
comportamientos más realistas, escalables y generalizables. El presente proyecto se enmarca en esta línea, buscando explorar nuevas formas de entrenamiento
y evaluación de sistemas pastores y rebaños mediante técnicas de inteligencia computacional y simulación multiagente.

Tomamos como principales referencias los trabajos de \citet{NapolitanoShepherding} y \citet{CovoneShepherding}, debido a que ambos aplican modelos de inteligencia computacional y son los trabajos más citados respecto al problema de pastoreo. En dichos trabajos, los agentes pastores aprenden estrategias cooperativas a partir de recomensas compartidas y observaciones parciales del entorno, logrando comportamientos de collecting y driving \citep{Strombom} sin supervisión.

En nuestra propuesta buscamos explorar un enfoque alternativo basado en algorítmos genéticos (GA) para el entrenamiento. Los métodos evolutivos resultan adecuados para entornos donde las políticas de control no pueden derivarse analíticamente y se deben explorar múltiples comportamientos emergentes. En lugar de aprender a través de la maximización de una recompensa acumulada, los agentes evolucionarán guiados por una función de aptitud inspirada en la función de recompensa de \citet{NapolitanoShepherding} que mida el desempeño del pastoreo.

Esta función de aptitud considerará métricas tales como:
\begin{itemize}
    \item La distancia promedio entre el rebaño y el objetivo final.
    \item El grado de dispersión del grupo durante la tarea.
    \item El tiempo total necesario para completar el pastoreo.
    \item Penalizaciones por comportamientos desorganizados o pérdida de individuos.
\end{itemize}

El entrenamiento se llevará a cabo en un entorno simulado donde tanto los agentes pastores como el rebaño interactúan bajo reglas de movimiento inspiradas en el modelo de \citet{Strombom}. A partir de estas reglas, los agentes pastores podrán aprender dinámicas de control emergentes efectivas, similares a las estrategias descritas en los trabajos citados \citep{Strombom} \citep{NapolitanoShepherding} \citep{CovoneShepherding}.

Una de las motivaciones para utilizar algorítmos genéticos es su capacidad para explorar espacios de búsqueda altamente no-lineales y multimodales, lo que podría facilitar la aparición de comportamientos cooperativos espontáneos sin requerir estructuras de control jerárquica ni el diseño de recomensas complejas. Además, este enfoque permitiría ampliar el estudio a escenarios más amplios de coevolución, como la interacción con depredadores.

Para la evaluación del desempeño se utilizarán métricas relacionadas con el éxito del pastoreo como son:
\begin{itemize}
    \item Porcentaje de individuos guiados correctamente.
    \item Tiempo de convergencia
    \item Estabilidad del rebaño
\end{itemize}

La observación del comportamiento en tiempo real se desarrollará sobre entornos de simulación bidimensionales inspirados en el trabajo de \citet{SankarShepherding} y los resultados se visualizarán en entornos similares a los de \citet{NapolitanoShepherding} y \citet{CovoneShepherding} para permitir una comparación directa con sus resultados.

Este enfoque busca evaluar la viabilidad de los métodos evolutivos como alternativa o complemento al aprendizaje por refuerzo en el problema del pastoreo multiagnete, contribuyendo al estudio de estrategias cooperativas dentro de entornos simulados.

\bibliographystyle{unsrtnat}
\bibliography{refs}
\end{document}

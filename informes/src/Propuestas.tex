\documentclass[final]{article}
\usepackage[numbers,sort&compress]{natbib} % formato de citas
\usepackage{neurips}            % estilo de la conferencia NeurIPS
\usepackage[spanish]{babel}     % definiciones para el español
\usepackage{graphicx}           % para insertar figuras
\usepackage{amsmath,amssymb}    % más símbolos útiles 
%\usepackage{todonotes}          % notas en el documento
\usepackage[hidelinks]{hyperref}% para los links
\usepackage[ruled,vlined,lined,linesnumbered,onelanguage]{algorithm2e} % algoritmos
\usepackage{multirow}           % algunas mejoras para las tablas
\usepackage{booktabs}


\begin{document}
\def\tablename{Tabla}          % (sino les pone Cuadro)

\title{Proyecto final creativo:\\
    Propuestas}

\author{G. Cagnola, F. G. Hergenreder y L. Saurin \\
        Facultad de Ingeniería y Ciencias Hídricas,
        Universidad Nacional del Litoral \\
        Instituto de Investigación en Señales, Sistemas e Inteligencia Computacional, UNL-CONICET}

\maketitle

\begin{abstract}
Breve descripción de las propuestas para el Proyecto Final Creativo.
\end{abstract}

%==========================================================
%==========================================================
\section{Clasificación de señales EEG y EMG para el control de una interfaz cerebro-computadora mediante técnicas de inteligencia computacional}
Implementar y comparar diferentes algoritmos de inteligencia computacional para la clasificación de señales EEG (actividad cerebral) y EMG (actividad muscular) orientadas al control de una BCI (interfaz cerebro-computadora). De esa forma lograr distinguir estas mentales/motores.

\section{Entrenamiento de colonias en entorno competitivos-colaborativos con métodos evolutivos para el estudio de estrategias de supervivencia colectiva emergentes}
Entrenar y enfrentar colonias de agentes con distintas técnicas y parámetros de evolución en un entorno con recursos limitados que promuevan la colaboración entre agentes de la misma colonia.

\section{Coevolución de sistema multiagente para comportamiento emergente rebaño-pastor}
Entrenar un rebaño de ovejas para aprender un comportamiento evasivo hacia los lobos. Luego (o quizás en simultaneo) entrenar uno o múltiples perros de pastoreo para aprender a guiar el rebaño hacia un objetivo.

\section{Simulador de tráfico impulsado por inteligencia artificial para ensayo de planificación urbana}
Entrenar un sistema multiagente para la simulación de tráfico. El objetivo es poder realizar pruebas de "rendimiento" y comparación en planificación de calles y/o control de semáforos.

\section{Detector de mentiras mediante telemetría de smartwatch}
Entrenar un modelo que detecte mentiras en tiempo real mediante señales de smartwatch como puede ser el pulso.

\section{Clasificación de acordes y generación de secuencias musicales}
Uso de mapas auto-organizativos para clasificación de acordes y generación de secuencias musicales mediantes MLP con RBF.

\section{Entrenamiento de agente para juego RTS Planet Wars mediante métodos evolutivos}
Entrenar mediante métodos evolutivos agentes que jueguen el juego Planet Wars\footnote{https://github.com/SimonLucas/planet-wars-rts}, propuesto como desafío en la GECCO 2025 e IEEE 2025 Conference on Games.

\section{Generación procedural de poblados en Minecraft}
Desarrollo de un sistema de generación procedural de aldeas en Minecraft, tomando como contexto el desafío de AI Settlement Generation Challenge\footnote{https://gendesignmc.wikidot.com/start}.

\end{document}
